\documentclass[11pt]{article}

    \usepackage[breakable]{tcolorbox}
    \usepackage{parskip} % Stop auto-indenting (to mimic markdown behaviour)
    
    \usepackage{iftex}
    \ifPDFTeX
    	\usepackage[T1]{fontenc}
    	\usepackage{mathpazo}
    \else
    	\usepackage{fontspec}
    \fi

    % Basic figure setup, for now with no caption control since it's done
    % automatically by Pandoc (which extracts ![](path) syntax from Markdown).
    \usepackage{graphicx}
    % Maintain compatibility with old templates. Remove in nbconvert 6.0
    \let\Oldincludegraphics\includegraphics
    % Ensure that by default, figures have no caption (until we provide a
    % proper Figure object with a Caption API and a way to capture that
    % in the conversion process - todo).
    \usepackage{caption}
    \DeclareCaptionFormat{nocaption}{}
    \captionsetup{format=nocaption,aboveskip=0pt,belowskip=0pt}

    \usepackage{float}
    \floatplacement{figure}{H} % forces figures to be placed at the correct location
    \usepackage{xcolor} % Allow colors to be defined
    \usepackage{enumerate} % Needed for markdown enumerations to work
    \usepackage{geometry} % Used to adjust the document margins
    \usepackage{amsmath} % Equations
    \usepackage{amssymb} % Equations
    \usepackage{textcomp} % defines textquotesingle
    % Hack from http://tex.stackexchange.com/a/47451/13684:
    \AtBeginDocument{%
        \def\PYZsq{\textquotesingle}% Upright quotes in Pygmentized code
    }
    \usepackage{upquote} % Upright quotes for verbatim code
    \usepackage{eurosym} % defines \euro
    \usepackage[mathletters]{ucs} % Extended unicode (utf-8) support
    \usepackage{fancyvrb} % verbatim replacement that allows latex
    \usepackage{grffile} % extends the file name processing of package graphics 
                         % to support a larger range
    \makeatletter % fix for old versions of grffile with XeLaTeX
    \@ifpackagelater{grffile}{2019/11/01}
    {
      % Do nothing on new versions
    }
    {
      \def\Gread@@xetex#1{%
        \IfFileExists{"\Gin@base".bb}%
        {\Gread@eps{\Gin@base.bb}}%
        {\Gread@@xetex@aux#1}%
      }
    }
    \makeatother
    \usepackage[Export]{adjustbox} % Used to constrain images to a maximum size
    \adjustboxset{max size={0.9\linewidth}{0.9\paperheight}}

    % The hyperref package gives us a pdf with properly built
    % internal navigation ('pdf bookmarks' for the table of contents,
    % internal cross-reference links, web links for URLs, etc.)
    \usepackage{hyperref}
    % The default LaTeX title has an obnoxious amount of whitespace. By default,
    % titling removes some of it. It also provides customization options.
    \usepackage{titling}
    \usepackage{longtable} % longtable support required by pandoc >1.10
    \usepackage{booktabs}  % table support for pandoc > 1.12.2
    \usepackage[inline]{enumitem} % IRkernel/repr support (it uses the enumerate* environment)
    \usepackage[normalem]{ulem} % ulem is needed to support strikethroughs (\sout)
                                % normalem makes italics be italics, not underlines
    \usepackage{mathrsfs}
    

    
    % Colors for the hyperref package
    \definecolor{urlcolor}{rgb}{0,.145,.698}
    \definecolor{linkcolor}{rgb}{.71,0.21,0.01}
    \definecolor{citecolor}{rgb}{.12,.54,.11}

    % ANSI colors
    \definecolor{ansi-black}{HTML}{3E424D}
    \definecolor{ansi-black-intense}{HTML}{282C36}
    \definecolor{ansi-red}{HTML}{E75C58}
    \definecolor{ansi-red-intense}{HTML}{B22B31}
    \definecolor{ansi-green}{HTML}{00A250}
    \definecolor{ansi-green-intense}{HTML}{007427}
    \definecolor{ansi-yellow}{HTML}{DDB62B}
    \definecolor{ansi-yellow-intense}{HTML}{B27D12}
    \definecolor{ansi-blue}{HTML}{208FFB}
    \definecolor{ansi-blue-intense}{HTML}{0065CA}
    \definecolor{ansi-magenta}{HTML}{D160C4}
    \definecolor{ansi-magenta-intense}{HTML}{A03196}
    \definecolor{ansi-cyan}{HTML}{60C6C8}
    \definecolor{ansi-cyan-intense}{HTML}{258F8F}
    \definecolor{ansi-white}{HTML}{C5C1B4}
    \definecolor{ansi-white-intense}{HTML}{A1A6B2}
    \definecolor{ansi-default-inverse-fg}{HTML}{FFFFFF}
    \definecolor{ansi-default-inverse-bg}{HTML}{000000}

    % common color for the border for error outputs.
    \definecolor{outerrorbackground}{HTML}{FFDFDF}

    % commands and environments needed by pandoc snippets
    % extracted from the output of `pandoc -s`
    \providecommand{\tightlist}{%
      \setlength{\itemsep}{0pt}\setlength{\parskip}{0pt}}
    \DefineVerbatimEnvironment{Highlighting}{Verbatim}{commandchars=\\\{\}}
    % Add ',fontsize=\small' for more characters per line
    \newenvironment{Shaded}{}{}
    \newcommand{\KeywordTok}[1]{\textcolor[rgb]{0.00,0.44,0.13}{\textbf{{#1}}}}
    \newcommand{\DataTypeTok}[1]{\textcolor[rgb]{0.56,0.13,0.00}{{#1}}}
    \newcommand{\DecValTok}[1]{\textcolor[rgb]{0.25,0.63,0.44}{{#1}}}
    \newcommand{\BaseNTok}[1]{\textcolor[rgb]{0.25,0.63,0.44}{{#1}}}
    \newcommand{\FloatTok}[1]{\textcolor[rgb]{0.25,0.63,0.44}{{#1}}}
    \newcommand{\CharTok}[1]{\textcolor[rgb]{0.25,0.44,0.63}{{#1}}}
    \newcommand{\StringTok}[1]{\textcolor[rgb]{0.25,0.44,0.63}{{#1}}}
    \newcommand{\CommentTok}[1]{\textcolor[rgb]{0.38,0.63,0.69}{\textit{{#1}}}}
    \newcommand{\OtherTok}[1]{\textcolor[rgb]{0.00,0.44,0.13}{{#1}}}
    \newcommand{\AlertTok}[1]{\textcolor[rgb]{1.00,0.00,0.00}{\textbf{{#1}}}}
    \newcommand{\FunctionTok}[1]{\textcolor[rgb]{0.02,0.16,0.49}{{#1}}}
    \newcommand{\RegionMarkerTok}[1]{{#1}}
    \newcommand{\ErrorTok}[1]{\textcolor[rgb]{1.00,0.00,0.00}{\textbf{{#1}}}}
    \newcommand{\NormalTok}[1]{{#1}}
    
    % Additional commands for more recent versions of Pandoc
    \newcommand{\ConstantTok}[1]{\textcolor[rgb]{0.53,0.00,0.00}{{#1}}}
    \newcommand{\SpecialCharTok}[1]{\textcolor[rgb]{0.25,0.44,0.63}{{#1}}}
    \newcommand{\VerbatimStringTok}[1]{\textcolor[rgb]{0.25,0.44,0.63}{{#1}}}
    \newcommand{\SpecialStringTok}[1]{\textcolor[rgb]{0.73,0.40,0.53}{{#1}}}
    \newcommand{\ImportTok}[1]{{#1}}
    \newcommand{\DocumentationTok}[1]{\textcolor[rgb]{0.73,0.13,0.13}{\textit{{#1}}}}
    \newcommand{\AnnotationTok}[1]{\textcolor[rgb]{0.38,0.63,0.69}{\textbf{\textit{{#1}}}}}
    \newcommand{\CommentVarTok}[1]{\textcolor[rgb]{0.38,0.63,0.69}{\textbf{\textit{{#1}}}}}
    \newcommand{\VariableTok}[1]{\textcolor[rgb]{0.10,0.09,0.49}{{#1}}}
    \newcommand{\ControlFlowTok}[1]{\textcolor[rgb]{0.00,0.44,0.13}{\textbf{{#1}}}}
    \newcommand{\OperatorTok}[1]{\textcolor[rgb]{0.40,0.40,0.40}{{#1}}}
    \newcommand{\BuiltInTok}[1]{{#1}}
    \newcommand{\ExtensionTok}[1]{{#1}}
    \newcommand{\PreprocessorTok}[1]{\textcolor[rgb]{0.74,0.48,0.00}{{#1}}}
    \newcommand{\AttributeTok}[1]{\textcolor[rgb]{0.49,0.56,0.16}{{#1}}}
    \newcommand{\InformationTok}[1]{\textcolor[rgb]{0.38,0.63,0.69}{\textbf{\textit{{#1}}}}}
    \newcommand{\WarningTok}[1]{\textcolor[rgb]{0.38,0.63,0.69}{\textbf{\textit{{#1}}}}}
    
    
    % Define a nice break command that doesn't care if a line doesn't already
    % exist.
    \def\br{\hspace*{\fill} \\* }
    % Math Jax compatibility definitions
    \def\gt{>}
    \def\lt{<}
    \let\Oldtex\TeX
    \let\Oldlatex\LaTeX
    \renewcommand{\TeX}{\textrm{\Oldtex}}
    \renewcommand{\LaTeX}{\textrm{\Oldlatex}}
    % Document parameters
    % Document title
    \title{Algorithms\_II\_Lab-02}
    
    
    
    
    
% Pygments definitions
\makeatletter
\def\PY@reset{\let\PY@it=\relax \let\PY@bf=\relax%
    \let\PY@ul=\relax \let\PY@tc=\relax%
    \let\PY@bc=\relax \let\PY@ff=\relax}
\def\PY@tok#1{\csname PY@tok@#1\endcsname}
\def\PY@toks#1+{\ifx\relax#1\empty\else%
    \PY@tok{#1}\expandafter\PY@toks\fi}
\def\PY@do#1{\PY@bc{\PY@tc{\PY@ul{%
    \PY@it{\PY@bf{\PY@ff{#1}}}}}}}
\def\PY#1#2{\PY@reset\PY@toks#1+\relax+\PY@do{#2}}

\expandafter\def\csname PY@tok@w\endcsname{\def\PY@tc##1{\textcolor[rgb]{0.73,0.73,0.73}{##1}}}
\expandafter\def\csname PY@tok@c\endcsname{\let\PY@it=\textit\def\PY@tc##1{\textcolor[rgb]{0.25,0.50,0.50}{##1}}}
\expandafter\def\csname PY@tok@cp\endcsname{\def\PY@tc##1{\textcolor[rgb]{0.74,0.48,0.00}{##1}}}
\expandafter\def\csname PY@tok@k\endcsname{\let\PY@bf=\textbf\def\PY@tc##1{\textcolor[rgb]{0.00,0.50,0.00}{##1}}}
\expandafter\def\csname PY@tok@kp\endcsname{\def\PY@tc##1{\textcolor[rgb]{0.00,0.50,0.00}{##1}}}
\expandafter\def\csname PY@tok@kt\endcsname{\def\PY@tc##1{\textcolor[rgb]{0.69,0.00,0.25}{##1}}}
\expandafter\def\csname PY@tok@o\endcsname{\def\PY@tc##1{\textcolor[rgb]{0.40,0.40,0.40}{##1}}}
\expandafter\def\csname PY@tok@ow\endcsname{\let\PY@bf=\textbf\def\PY@tc##1{\textcolor[rgb]{0.67,0.13,1.00}{##1}}}
\expandafter\def\csname PY@tok@nb\endcsname{\def\PY@tc##1{\textcolor[rgb]{0.00,0.50,0.00}{##1}}}
\expandafter\def\csname PY@tok@nf\endcsname{\def\PY@tc##1{\textcolor[rgb]{0.00,0.00,1.00}{##1}}}
\expandafter\def\csname PY@tok@nc\endcsname{\let\PY@bf=\textbf\def\PY@tc##1{\textcolor[rgb]{0.00,0.00,1.00}{##1}}}
\expandafter\def\csname PY@tok@nn\endcsname{\let\PY@bf=\textbf\def\PY@tc##1{\textcolor[rgb]{0.00,0.00,1.00}{##1}}}
\expandafter\def\csname PY@tok@ne\endcsname{\let\PY@bf=\textbf\def\PY@tc##1{\textcolor[rgb]{0.82,0.25,0.23}{##1}}}
\expandafter\def\csname PY@tok@nv\endcsname{\def\PY@tc##1{\textcolor[rgb]{0.10,0.09,0.49}{##1}}}
\expandafter\def\csname PY@tok@no\endcsname{\def\PY@tc##1{\textcolor[rgb]{0.53,0.00,0.00}{##1}}}
\expandafter\def\csname PY@tok@nl\endcsname{\def\PY@tc##1{\textcolor[rgb]{0.63,0.63,0.00}{##1}}}
\expandafter\def\csname PY@tok@ni\endcsname{\let\PY@bf=\textbf\def\PY@tc##1{\textcolor[rgb]{0.60,0.60,0.60}{##1}}}
\expandafter\def\csname PY@tok@na\endcsname{\def\PY@tc##1{\textcolor[rgb]{0.49,0.56,0.16}{##1}}}
\expandafter\def\csname PY@tok@nt\endcsname{\let\PY@bf=\textbf\def\PY@tc##1{\textcolor[rgb]{0.00,0.50,0.00}{##1}}}
\expandafter\def\csname PY@tok@nd\endcsname{\def\PY@tc##1{\textcolor[rgb]{0.67,0.13,1.00}{##1}}}
\expandafter\def\csname PY@tok@s\endcsname{\def\PY@tc##1{\textcolor[rgb]{0.73,0.13,0.13}{##1}}}
\expandafter\def\csname PY@tok@sd\endcsname{\let\PY@it=\textit\def\PY@tc##1{\textcolor[rgb]{0.73,0.13,0.13}{##1}}}
\expandafter\def\csname PY@tok@si\endcsname{\let\PY@bf=\textbf\def\PY@tc##1{\textcolor[rgb]{0.73,0.40,0.53}{##1}}}
\expandafter\def\csname PY@tok@se\endcsname{\let\PY@bf=\textbf\def\PY@tc##1{\textcolor[rgb]{0.73,0.40,0.13}{##1}}}
\expandafter\def\csname PY@tok@sr\endcsname{\def\PY@tc##1{\textcolor[rgb]{0.73,0.40,0.53}{##1}}}
\expandafter\def\csname PY@tok@ss\endcsname{\def\PY@tc##1{\textcolor[rgb]{0.10,0.09,0.49}{##1}}}
\expandafter\def\csname PY@tok@sx\endcsname{\def\PY@tc##1{\textcolor[rgb]{0.00,0.50,0.00}{##1}}}
\expandafter\def\csname PY@tok@m\endcsname{\def\PY@tc##1{\textcolor[rgb]{0.40,0.40,0.40}{##1}}}
\expandafter\def\csname PY@tok@gh\endcsname{\let\PY@bf=\textbf\def\PY@tc##1{\textcolor[rgb]{0.00,0.00,0.50}{##1}}}
\expandafter\def\csname PY@tok@gu\endcsname{\let\PY@bf=\textbf\def\PY@tc##1{\textcolor[rgb]{0.50,0.00,0.50}{##1}}}
\expandafter\def\csname PY@tok@gd\endcsname{\def\PY@tc##1{\textcolor[rgb]{0.63,0.00,0.00}{##1}}}
\expandafter\def\csname PY@tok@gi\endcsname{\def\PY@tc##1{\textcolor[rgb]{0.00,0.63,0.00}{##1}}}
\expandafter\def\csname PY@tok@gr\endcsname{\def\PY@tc##1{\textcolor[rgb]{1.00,0.00,0.00}{##1}}}
\expandafter\def\csname PY@tok@ge\endcsname{\let\PY@it=\textit}
\expandafter\def\csname PY@tok@gs\endcsname{\let\PY@bf=\textbf}
\expandafter\def\csname PY@tok@gp\endcsname{\let\PY@bf=\textbf\def\PY@tc##1{\textcolor[rgb]{0.00,0.00,0.50}{##1}}}
\expandafter\def\csname PY@tok@go\endcsname{\def\PY@tc##1{\textcolor[rgb]{0.53,0.53,0.53}{##1}}}
\expandafter\def\csname PY@tok@gt\endcsname{\def\PY@tc##1{\textcolor[rgb]{0.00,0.27,0.87}{##1}}}
\expandafter\def\csname PY@tok@err\endcsname{\def\PY@bc##1{\setlength{\fboxsep}{0pt}\fcolorbox[rgb]{1.00,0.00,0.00}{1,1,1}{\strut ##1}}}
\expandafter\def\csname PY@tok@kc\endcsname{\let\PY@bf=\textbf\def\PY@tc##1{\textcolor[rgb]{0.00,0.50,0.00}{##1}}}
\expandafter\def\csname PY@tok@kd\endcsname{\let\PY@bf=\textbf\def\PY@tc##1{\textcolor[rgb]{0.00,0.50,0.00}{##1}}}
\expandafter\def\csname PY@tok@kn\endcsname{\let\PY@bf=\textbf\def\PY@tc##1{\textcolor[rgb]{0.00,0.50,0.00}{##1}}}
\expandafter\def\csname PY@tok@kr\endcsname{\let\PY@bf=\textbf\def\PY@tc##1{\textcolor[rgb]{0.00,0.50,0.00}{##1}}}
\expandafter\def\csname PY@tok@bp\endcsname{\def\PY@tc##1{\textcolor[rgb]{0.00,0.50,0.00}{##1}}}
\expandafter\def\csname PY@tok@fm\endcsname{\def\PY@tc##1{\textcolor[rgb]{0.00,0.00,1.00}{##1}}}
\expandafter\def\csname PY@tok@vc\endcsname{\def\PY@tc##1{\textcolor[rgb]{0.10,0.09,0.49}{##1}}}
\expandafter\def\csname PY@tok@vg\endcsname{\def\PY@tc##1{\textcolor[rgb]{0.10,0.09,0.49}{##1}}}
\expandafter\def\csname PY@tok@vi\endcsname{\def\PY@tc##1{\textcolor[rgb]{0.10,0.09,0.49}{##1}}}
\expandafter\def\csname PY@tok@vm\endcsname{\def\PY@tc##1{\textcolor[rgb]{0.10,0.09,0.49}{##1}}}
\expandafter\def\csname PY@tok@sa\endcsname{\def\PY@tc##1{\textcolor[rgb]{0.73,0.13,0.13}{##1}}}
\expandafter\def\csname PY@tok@sb\endcsname{\def\PY@tc##1{\textcolor[rgb]{0.73,0.13,0.13}{##1}}}
\expandafter\def\csname PY@tok@sc\endcsname{\def\PY@tc##1{\textcolor[rgb]{0.73,0.13,0.13}{##1}}}
\expandafter\def\csname PY@tok@dl\endcsname{\def\PY@tc##1{\textcolor[rgb]{0.73,0.13,0.13}{##1}}}
\expandafter\def\csname PY@tok@s2\endcsname{\def\PY@tc##1{\textcolor[rgb]{0.73,0.13,0.13}{##1}}}
\expandafter\def\csname PY@tok@sh\endcsname{\def\PY@tc##1{\textcolor[rgb]{0.73,0.13,0.13}{##1}}}
\expandafter\def\csname PY@tok@s1\endcsname{\def\PY@tc##1{\textcolor[rgb]{0.73,0.13,0.13}{##1}}}
\expandafter\def\csname PY@tok@mb\endcsname{\def\PY@tc##1{\textcolor[rgb]{0.40,0.40,0.40}{##1}}}
\expandafter\def\csname PY@tok@mf\endcsname{\def\PY@tc##1{\textcolor[rgb]{0.40,0.40,0.40}{##1}}}
\expandafter\def\csname PY@tok@mh\endcsname{\def\PY@tc##1{\textcolor[rgb]{0.40,0.40,0.40}{##1}}}
\expandafter\def\csname PY@tok@mi\endcsname{\def\PY@tc##1{\textcolor[rgb]{0.40,0.40,0.40}{##1}}}
\expandafter\def\csname PY@tok@il\endcsname{\def\PY@tc##1{\textcolor[rgb]{0.40,0.40,0.40}{##1}}}
\expandafter\def\csname PY@tok@mo\endcsname{\def\PY@tc##1{\textcolor[rgb]{0.40,0.40,0.40}{##1}}}
\expandafter\def\csname PY@tok@ch\endcsname{\let\PY@it=\textit\def\PY@tc##1{\textcolor[rgb]{0.25,0.50,0.50}{##1}}}
\expandafter\def\csname PY@tok@cm\endcsname{\let\PY@it=\textit\def\PY@tc##1{\textcolor[rgb]{0.25,0.50,0.50}{##1}}}
\expandafter\def\csname PY@tok@cpf\endcsname{\let\PY@it=\textit\def\PY@tc##1{\textcolor[rgb]{0.25,0.50,0.50}{##1}}}
\expandafter\def\csname PY@tok@c1\endcsname{\let\PY@it=\textit\def\PY@tc##1{\textcolor[rgb]{0.25,0.50,0.50}{##1}}}
\expandafter\def\csname PY@tok@cs\endcsname{\let\PY@it=\textit\def\PY@tc##1{\textcolor[rgb]{0.25,0.50,0.50}{##1}}}

\def\PYZbs{\char`\\}
\def\PYZus{\char`\_}
\def\PYZob{\char`\{}
\def\PYZcb{\char`\}}
\def\PYZca{\char`\^}
\def\PYZam{\char`\&}
\def\PYZlt{\char`\<}
\def\PYZgt{\char`\>}
\def\PYZsh{\char`\#}
\def\PYZpc{\char`\%}
\def\PYZdl{\char`\$}
\def\PYZhy{\char`\-}
\def\PYZsq{\char`\'}
\def\PYZdq{\char`\"}
\def\PYZti{\char`\~}
% for compatibility with earlier versions
\def\PYZat{@}
\def\PYZlb{[}
\def\PYZrb{]}
\makeatother


    % For linebreaks inside Verbatim environment from package fancyvrb. 
    \makeatletter
        \newbox\Wrappedcontinuationbox 
        \newbox\Wrappedvisiblespacebox 
        \newcommand*\Wrappedvisiblespace {\textcolor{red}{\textvisiblespace}} 
        \newcommand*\Wrappedcontinuationsymbol {\textcolor{red}{\llap{\tiny$\m@th\hookrightarrow$}}} 
        \newcommand*\Wrappedcontinuationindent {3ex } 
        \newcommand*\Wrappedafterbreak {\kern\Wrappedcontinuationindent\copy\Wrappedcontinuationbox} 
        % Take advantage of the already applied Pygments mark-up to insert 
        % potential linebreaks for TeX processing. 
        %        {, <, #, %, $, ' and ": go to next line. 
        %        _, }, ^, &, >, - and ~: stay at end of broken line. 
        % Use of \textquotesingle for straight quote. 
        \newcommand*\Wrappedbreaksatspecials {% 
            \def\PYGZus{\discretionary{\char`\_}{\Wrappedafterbreak}{\char`\_}}% 
            \def\PYGZob{\discretionary{}{\Wrappedafterbreak\char`\{}{\char`\{}}% 
            \def\PYGZcb{\discretionary{\char`\}}{\Wrappedafterbreak}{\char`\}}}% 
            \def\PYGZca{\discretionary{\char`\^}{\Wrappedafterbreak}{\char`\^}}% 
            \def\PYGZam{\discretionary{\char`\&}{\Wrappedafterbreak}{\char`\&}}% 
            \def\PYGZlt{\discretionary{}{\Wrappedafterbreak\char`\<}{\char`\<}}% 
            \def\PYGZgt{\discretionary{\char`\>}{\Wrappedafterbreak}{\char`\>}}% 
            \def\PYGZsh{\discretionary{}{\Wrappedafterbreak\char`\#}{\char`\#}}% 
            \def\PYGZpc{\discretionary{}{\Wrappedafterbreak\char`\%}{\char`\%}}% 
            \def\PYGZdl{\discretionary{}{\Wrappedafterbreak\char`\$}{\char`\$}}% 
            \def\PYGZhy{\discretionary{\char`\-}{\Wrappedafterbreak}{\char`\-}}% 
            \def\PYGZsq{\discretionary{}{\Wrappedafterbreak\textquotesingle}{\textquotesingle}}% 
            \def\PYGZdq{\discretionary{}{\Wrappedafterbreak\char`\"}{\char`\"}}% 
            \def\PYGZti{\discretionary{\char`\~}{\Wrappedafterbreak}{\char`\~}}% 
        } 
        % Some characters . , ; ? ! / are not pygmentized. 
        % This macro makes them "active" and they will insert potential linebreaks 
        \newcommand*\Wrappedbreaksatpunct {% 
            \lccode`\~`\.\lowercase{\def~}{\discretionary{\hbox{\char`\.}}{\Wrappedafterbreak}{\hbox{\char`\.}}}% 
            \lccode`\~`\,\lowercase{\def~}{\discretionary{\hbox{\char`\,}}{\Wrappedafterbreak}{\hbox{\char`\,}}}% 
            \lccode`\~`\;\lowercase{\def~}{\discretionary{\hbox{\char`\;}}{\Wrappedafterbreak}{\hbox{\char`\;}}}% 
            \lccode`\~`\:\lowercase{\def~}{\discretionary{\hbox{\char`\:}}{\Wrappedafterbreak}{\hbox{\char`\:}}}% 
            \lccode`\~`\?\lowercase{\def~}{\discretionary{\hbox{\char`\?}}{\Wrappedafterbreak}{\hbox{\char`\?}}}% 
            \lccode`\~`\!\lowercase{\def~}{\discretionary{\hbox{\char`\!}}{\Wrappedafterbreak}{\hbox{\char`\!}}}% 
            \lccode`\~`\/\lowercase{\def~}{\discretionary{\hbox{\char`\/}}{\Wrappedafterbreak}{\hbox{\char`\/}}}% 
            \catcode`\.\active
            \catcode`\,\active 
            \catcode`\;\active
            \catcode`\:\active
            \catcode`\?\active
            \catcode`\!\active
            \catcode`\/\active 
            \lccode`\~`\~ 	
        }
    \makeatother

    \let\OriginalVerbatim=\Verbatim
    \makeatletter
    \renewcommand{\Verbatim}[1][1]{%
        %\parskip\z@skip
        \sbox\Wrappedcontinuationbox {\Wrappedcontinuationsymbol}%
        \sbox\Wrappedvisiblespacebox {\FV@SetupFont\Wrappedvisiblespace}%
        \def\FancyVerbFormatLine ##1{\hsize\linewidth
            \vtop{\raggedright\hyphenpenalty\z@\exhyphenpenalty\z@
                \doublehyphendemerits\z@\finalhyphendemerits\z@
                \strut ##1\strut}%
        }%
        % If the linebreak is at a space, the latter will be displayed as visible
        % space at end of first line, and a continuation symbol starts next line.
        % Stretch/shrink are however usually zero for typewriter font.
        \def\FV@Space {%
            \nobreak\hskip\z@ plus\fontdimen3\font minus\fontdimen4\font
            \discretionary{\copy\Wrappedvisiblespacebox}{\Wrappedafterbreak}
            {\kern\fontdimen2\font}%
        }%
        
        % Allow breaks at special characters using \PYG... macros.
        \Wrappedbreaksatspecials
        % Breaks at punctuation characters . , ; ? ! and / need catcode=\active 	
        \OriginalVerbatim[#1,codes*=\Wrappedbreaksatpunct]%
    }
    \makeatother

    % Exact colors from NB
    \definecolor{incolor}{HTML}{303F9F}
    \definecolor{outcolor}{HTML}{D84315}
    \definecolor{cellborder}{HTML}{CFCFCF}
    \definecolor{cellbackground}{HTML}{F7F7F7}
    
    % prompt
    \makeatletter
    \newcommand{\boxspacing}{\kern\kvtcb@left@rule\kern\kvtcb@boxsep}
    \makeatother
    \newcommand{\prompt}[4]{
        {\ttfamily\llap{{\color{#2}[#3]:\hspace{3pt}#4}}\vspace{-\baselineskip}}
    }
    

    
    % Prevent overflowing lines due to hard-to-break entities
    \sloppy 
    % Setup hyperref package
    \hypersetup{
      breaklinks=true,  % so long urls are correctly broken across lines
      colorlinks=true,
      urlcolor=urlcolor,
      linkcolor=linkcolor,
      citecolor=citecolor,
      }
    % Slightly bigger margins than the latex defaults
    
    \geometry{verbose,tmargin=1in,bmargin=1in,lmargin=1in,rmargin=1in}
    
    

\begin{document}
    
    \maketitle
    
    

    
    \begin{tcolorbox}[breakable, size=fbox, boxrule=1pt, pad at break*=1mm,colback=cellbackground, colframe=cellborder]
\prompt{In}{incolor}{47}{\boxspacing}
\begin{Verbatim}[commandchars=\\\{\}]
\PY{n}{params} \PY{o}{=} \PY{n+nf}{seq}\PY{p}{(}\PY{l+m}{0}\PY{p}{,}\PY{l+m}{1}\PY{p}{,}\PY{l+m}{0.001}\PY{p}{)}

\PY{n+nf}{lines}\PY{p}{(}\PY{n}{params}\PY{p}{,}\PY{n}{dens}\PY{p}{)}
\end{Verbatim}
\end{tcolorbox}

    \begin{Verbatim}[commandchars=\\\{\}, frame=single, framerule=2mm, rulecolor=\color{outerrorbackground}]
Error in plot.xy(xy.coords(x, y), type = type, {\ldots}): plot.new has not been called yet
Traceback:

1. lines(params, dens)
2. lines.default(params, dens)
3. plot.xy(xy.coords(x, y), type = type, {\ldots})
    \end{Verbatim}

    \hypertarget{lab02}{%
\subsection{Lab02}\label{lab02}}

Class will focus on learning the probability of heads from a series of
coin flips using the maximum likelihood approach. We will learn how to
simulate a series of coin flips, program the loglikelihood function into
R, optimize this likelihood and approximate the distribution of maximum
likelihood estimates for the probability of heads using the Fisher
Information Matrix.

    \hypertarget{simulate-a-coin-flip-using-the-rbinom-function}{%
\subsubsection{\texorpdfstring{Simulate a coin flip using the
\texttt{rbinom}
function}{Simulate a coin flip using the rbinom function}}\label{simulate-a-coin-flip-using-the-rbinom-function}}

We will simulate a coinflip using the \texttt{rbinom} fucntion. Recall
that the \href{}{Binomial distribution or Binom(N,\(\theta\))} assigns
probabilities to the values 0,1,2,\ldots,N.

The probability mass function (pmf) for the Binomial distribution is
\begin{align}
    p(x|N,\theta) = \binom{N}{x} \theta^{x}(1-\theta)^{N-x}
\end{align} and the pmf for the Bernoulli is \(\theta\) when \(x=1\) and
\(1-\theta\) when \(x=0\). We can write this pmf in a single line like
this \begin{align}
    p(x|\theta) = \theta^{x}(1-\theta)^{1-x}
\end{align}

By setting the Binomial distribution parameter \(N=1\), we have the
following pmf \begin{align}
    p(x|N=1,\theta) &= \binom{1}{x}\theta^{x}(1-\theta)^{1-x} \\ 
                    &= \theta^{x}(1-\theta)^{1-x}
\end{align} where \(\binom{1}{x}\) counts the number of ways to return 1
item (which is equal to one).

This means we can simulate values from a \(\text{Bern}(\theta)\)
distribution by simulating from a \(\text{Binom}(1,\theta)\)
distribution. In R, we can simulate values from a Binomial distribution
using the \texttt{rbinom} function,

    \begin{tcolorbox}[breakable, size=fbox, boxrule=1pt, pad at break*=1mm,colback=cellbackground, colframe=cellborder]
\prompt{In}{incolor}{ }{\boxspacing}
\begin{Verbatim}[commandchars=\\\{\}]
\PY{n+nf}{rbinom}\PY{p}{(}\PY{n}{size}\PY{o}{=}\PY{l+m}{10}\PY{p}{,}\PY{n}{p}\PY{o}{=}\PY{l+m}{0.25}\PY{p}{,}\PY{n}{n}\PY{o}{=}\PY{l+m}{30}\PY{p}{)}
\end{Verbatim}
\end{tcolorbox}

    where the input \textbf{size} corresponds to the parameter \(N\), the
input \(p\) corresponds to \(\theta\), and \(n\) controls the number of
values you wish to generate.

    Lets create a function that simulates \(N\) coin flips with a
probability \(p\) os heads and probability \(1-p\) of tails.

    \begin{tcolorbox}[breakable, size=fbox, boxrule=1pt, pad at break*=1mm,colback=cellbackground, colframe=cellborder]
\prompt{In}{incolor}{ }{\boxspacing}
\begin{Verbatim}[commandchars=\\\{\}]
\PY{n}{simulateNflips} \PY{o}{=} \PY{n+nf}{function}\PY{p}{(}\PY{n}{numflips}\PY{p}{,}\PY{n}{p}\PY{p}{)}\PY{p}{\PYZob{}}
  \PY{n}{samples} \PY{o}{=} \PY{n+nf}{rbinom}\PY{p}{(}\PY{n}{n}\PY{o}{=}\PY{n}{numflips}\PY{p}{,}\PY{n}{size}\PY{o}{=}\PY{l+m}{1}\PY{p}{,}\PY{n}{p}\PY{o}{=}\PY{n}{p}\PY{p}{)}
  \PY{n+nf}{return}\PY{p}{(}\PY{n}{samples}\PY{p}{)}
\PY{p}{\PYZcb{}}
\end{Verbatim}
\end{tcolorbox}

    \begin{tcolorbox}[breakable, size=fbox, boxrule=1pt, pad at break*=1mm,colback=cellbackground, colframe=cellborder]
\prompt{In}{incolor}{ }{\boxspacing}
\begin{Verbatim}[commandchars=\\\{\}]
\PY{n+nf}{simulateNflips}\PY{p}{(}\PY{l+m}{100}\PY{p}{,}\PY{l+m}{0.45}\PY{p}{)}
\end{Verbatim}
\end{tcolorbox}

    Now that we have a convienant function, lets simulate 1,000 coin flips
with a prob of 0.25 for heads and store this vector in a variable called
\textbf{samples}.

    \begin{tcolorbox}[breakable, size=fbox, boxrule=1pt, pad at break*=1mm,colback=cellbackground, colframe=cellborder]
\prompt{In}{incolor}{ }{\boxspacing}
\begin{Verbatim}[commandchars=\\\{\}]
\PY{n}{samples} \PY{o}{=} \PY{n+nf}{simulateNflips}\PY{p}{(}\PY{l+m}{1000}\PY{p}{,}\PY{l+m}{0.25}\PY{p}{)}
\end{Verbatim}
\end{tcolorbox}

    \hypertarget{the-loglikelihood-function-for-the-bernoulli-distribution}{%
\subsubsection{The loglikelihood function for the Bernoulli
distribution}\label{the-loglikelihood-function-for-the-bernoulli-distribution}}

\hypertarget{for-a-single-sample}{%
\paragraph{For a single sample}\label{for-a-single-sample}}

For a single data point, if we knew the value of \(\theta\) we would
assign the foloowing probability \begin{align}
    p(x|\theta) = \theta^{x}(1-\theta)^{1-x} 
\end{align} In other words, if we saw the value \(1\) we would assign
that the probability \(\theta\) and if we saw \(0\) we would assign that
value a \(1-\theta\) probability of occuring.

However, for a given data point we can view the above probability as a
function of \(\theta\) \begin{align}
    g(\theta) = p(x|\theta) = \theta^{x}(1-\theta)^{1-x} 
\end{align}

This function is called the likelihood. The log likelihood is then

\begin{align}
    \log(g(\theta)) &= \log[ \theta^{x}(1-\theta)^{1-x} ] \\
                    &= x\log(\theta) + (1-x)\log(1-\theta)
\end{align}

\hypertarget{for-n-samples}{%
\paragraph{For N samples}\label{for-n-samples}}

For \(N\) data points that we represent as \(x_{1},x_{2},\cdots,x_{N}\)
where \(x_{i}\) is a 1 or 0, the log likelihood becomes

\begin{align}
    p(x_{1},x_{2},\cdots,x_{N}|\theta) = \left[ \theta^{x_{1}}(1-\theta)^{1-x_{1}} \right] \left[ \theta^{x_{2}}(1-\theta)^{1-x_{2}} \right] \cdots \left[ \theta^{x_{N}}(1-\theta)^{1-x_{N}} \right]
\end{align}

\begin{align}
    \log \left[g(\theta)\right] &= \log \left\{ \left[ \theta^{x_{1}}(1-\theta)^{1-x_{1}} \right] \left[ \theta^{x_{2}}(1-\theta)^{1-x_{2}} \right] \cdots \left[ \theta^{x_{N}}(1-\theta)^{1-x_{N}} \right] \right \} \\
    &= \sum_{i=1}^{N} x_{i} \log(\theta) + (1-x_{i}) \log(1-\theta) \\
    &= \log(\theta) \sum_{i=1}^{N} x_{i}  + \log(1-\theta) \left(N-\sum_{i=1}^{N} x_{i}\right)  \\
\end{align}

    Lets spend time implementing a function called \texttt{loglikelihood}
that takes as input two variables: a single number between 0 and 1
called \texttt{theta} and a vector of 0s and 1s called \texttt{samples}.
This function will return the loglikelihood above.

    \begin{tcolorbox}[breakable, size=fbox, boxrule=1pt, pad at break*=1mm,colback=cellbackground, colframe=cellborder]
\prompt{In}{incolor}{ }{\boxspacing}
\begin{Verbatim}[commandchars=\\\{\}]
\PY{n}{loglikelihood} \PY{o}{=} \PY{n+nf}{function}\PY{p}{(}\PY{n}{theta}\PY{p}{,}\PY{n}{samples}\PY{p}{)}\PY{p}{\PYZob{}}
  \PY{n}{N} \PY{o}{=} \PY{n+nf}{length}\PY{p}{(}\PY{n}{samples}\PY{p}{)}
  \PY{n}{numberOfOnes}  \PY{o}{=} \PY{n+nf}{sum}\PY{p}{(}\PY{n}{samples}\PY{p}{)}
  \PY{n}{numberOfZeros} \PY{o}{=} \PY{n}{N}\PY{o}{\PYZhy{}}\PY{n}{numberOfOnes}
  \PY{n}{LL} \PY{o}{=} \PY{n}{numberOfOnes}\PY{o}{*}\PY{n+nf}{log}\PY{p}{(}\PY{n}{theta}\PY{p}{)} \PY{o}{+} \PY{n}{numberOfZeros}\PY{o}{*}\PY{n+nf}{log}\PY{p}{(}\PY{l+m}{1}\PY{o}{\PYZhy{}}\PY{n}{theta}\PY{p}{)}
  \PY{n+nf}{return}\PY{p}{(}\PY{n}{LL}\PY{p}{)}
\PY{p}{\PYZcb{}}
\end{Verbatim}
\end{tcolorbox}

    \hypertarget{visualizing-the-loglikelihood-function-for-a-sample-of-data}{%
\subsubsection{Visualizing the loglikelihood function for a sample of
data}\label{visualizing-the-loglikelihood-function-for-a-sample-of-data}}

We can plot the loglikelihood for values of theta from 0 to 1 by 0.01 to
get a sense of where the loglikelihood is highest. Because we simulated
our data \texttt{samples} using \(\theta=0.25\) we expect that the
loglikelihood will be highest around 0.25.

    \begin{tcolorbox}[breakable, size=fbox, boxrule=1pt, pad at break*=1mm,colback=cellbackground, colframe=cellborder]
\prompt{In}{incolor}{ }{\boxspacing}
\begin{Verbatim}[commandchars=\\\{\}]
\PY{n}{thetas} \PY{o}{=} \PY{n+nf}{seq}\PY{p}{(}\PY{l+m}{0}\PY{p}{,}\PY{l+m}{1}\PY{p}{,}\PY{l+m}{0.01}\PY{p}{)}
\PY{n}{LLs} \PY{o}{=} \PY{n+nf}{loglikelihood}\PY{p}{(}\PY{n}{thetas}\PY{p}{,}\PY{n}{samples}\PY{p}{)}
\PY{n+nf}{plot}\PY{p}{(}\PY{n}{thetas}\PY{p}{,}\PY{n}{LLs}\PY{p}{)}
\end{Verbatim}
\end{tcolorbox}

    \hypertarget{finding-the-maximum-of-the-loglikelihood-using-an-optimization-routine}{%
\subsubsection{Finding the maximum of the loglikelihood using an
optimization
routine}\label{finding-the-maximum-of-the-loglikelihood-using-an-optimization-routine}}

To find the \(\theta\) that maximizes the loglikelihood function we will
use the R function called
\href{https://www.rdocumentation.org/packages/stats/versions/3.6.2/topics/optimize}{\texttt{optimize}}.
The optimize function takes as input: a function of one variable that we
wish to optimize and an interval of potential values that will maximize
this function.

First, we need to give the \texttt{optimize} function our
\texttt{loglikelihood} fucntion. But we have a problem. The
loglikelihood function we created takes two inputs: \(\theta\) and
\texttt{samples}. We need to create a new function (lets call it
\texttt{LL}), often called a \textbf{wrapper}, that takes as input only
theta and returns the loglikehood.

    \begin{tcolorbox}[breakable, size=fbox, boxrule=1pt, pad at break*=1mm,colback=cellbackground, colframe=cellborder]
\prompt{In}{incolor}{ }{\boxspacing}
\begin{Verbatim}[commandchars=\\\{\}]
\PY{n}{LL} \PY{o}{=} \PY{n+nf}{function}\PY{p}{(}\PY{n}{theta}\PY{p}{)}\PY{p}{\PYZob{}}
  \PY{n+nf}{return}\PY{p}{(}\PY{n+nf}{loglikelihood}\PY{p}{(}\PY{n}{theta}\PY{p}{,}\PY{n}{samples}\PY{p}{)}\PY{p}{)}
\PY{p}{\PYZcb{}}
\end{Verbatim}
\end{tcolorbox}

    We can now call the optimize function and input LL. The second argument
is the interval of possible values that could optimize this
loglikelihood. The possible values of \(\theta\) are in the interval
from 0 to 1, and so we can provide the optimize function this inverval.
Finally, the default behavior of the optimize function is to find a
minimum value, not maximum. We can change this behavior by providing the
optimize function one optional input, maximum, that is either TRUE or
FALSE.

    \begin{tcolorbox}[breakable, size=fbox, boxrule=1pt, pad at break*=1mm,colback=cellbackground, colframe=cellborder]
\prompt{In}{incolor}{ }{\boxspacing}
\begin{Verbatim}[commandchars=\\\{\}]
\PY{n+nf}{optimize}\PY{p}{(}\PY{n}{LL}\PY{p}{,} \PY{n}{interval}\PY{o}{=}\PY{n+nf}{c}\PY{p}{(}\PY{l+m}{0}\PY{p}{,}\PY{l+m}{1}\PY{p}{)}\PY{p}{,}\PY{n}{maximum}\PY{o}{=}\PY{k+kc}{TRUE}\PY{p}{)}
\end{Verbatim}
\end{tcolorbox}

    The optimize function returns a new object we have not yet seen in R---a
\textbf{list}. This list contains the maximum \(\theta\) value and also
the loglikelihood value for this maximum \(\theta\) value.

    \hypertarget{the-list-an-r-programming-aside}{%
\subsubsection{The List (an R programming
aside)}\label{the-list-an-r-programming-aside}}

In R, a list can be used to store items of different types and reference
them with a key. To create a list, use the
\href{https://www.rdocumentation.org/packages/base/versions/3.6.2/topics/list}{list}
function and include as many R objects as you like.

For example, we can create a list called \texttt{L} that contains a a
key ``A'' that correpsonds to a string ``B'' and a key ``C'' that
corresponds to a vector. Keys need to be strings.

    \begin{tcolorbox}[breakable, size=fbox, boxrule=1pt, pad at break*=1mm,colback=cellbackground, colframe=cellborder]
\prompt{In}{incolor}{ }{\boxspacing}
\begin{Verbatim}[commandchars=\\\{\}]
\PY{n}{L} \PY{o}{=} \PY{n+nf}{list}\PY{p}{(}\PY{l+s}{\PYZdq{}}\PY{l+s}{A\PYZdq{}}\PY{o}{=}\PY{l+s}{\PYZdq{}}\PY{l+s}{B\PYZdq{}}\PY{p}{,}\PY{l+s}{\PYZdq{}}\PY{l+s}{C\PYZdq{}} \PY{o}{=} \PY{n+nf}{c}\PY{p}{(}\PY{l+m}{1}\PY{p}{,}\PY{l+m}{2}\PY{p}{,}\PY{l+m}{3}\PY{p}{)}\PY{p}{)}
\PY{n}{L}
\end{Verbatim}
\end{tcolorbox}

    Items in the list can be accessed by naming the list and enclosing two
sets of square brackets around a key.

    \begin{tcolorbox}[breakable, size=fbox, boxrule=1pt, pad at break*=1mm,colback=cellbackground, colframe=cellborder]
\prompt{In}{incolor}{ }{\boxspacing}
\begin{Verbatim}[commandchars=\\\{\}]
\PY{n}{L}\PY{p}{[[}\PY{l+s}{\PYZdq{}}\PY{l+s}{A\PYZdq{}}\PY{p}{]]}
\end{Verbatim}
\end{tcolorbox}

    \begin{tcolorbox}[breakable, size=fbox, boxrule=1pt, pad at break*=1mm,colback=cellbackground, colframe=cellborder]
\prompt{In}{incolor}{ }{\boxspacing}
\begin{Verbatim}[commandchars=\\\{\}]
\PY{n}{L}\PY{p}{[[}\PY{l+s}{\PYZdq{}}\PY{l+s}{C\PYZdq{}}\PY{p}{]]}
\end{Verbatim}
\end{tcolorbox}

    An alternative way to access a list is by using the \$ operator. Name
the list, type a \$, and then the key.

    \begin{tcolorbox}[breakable, size=fbox, boxrule=1pt, pad at break*=1mm,colback=cellbackground, colframe=cellborder]
\prompt{In}{incolor}{ }{\boxspacing}
\begin{Verbatim}[commandchars=\\\{\}]
\PY{n}{L}\PY{o}{\PYZdl{}}\PY{n}{C}
\end{Verbatim}
\end{tcolorbox}

    We now see that the optimize function returns a list of two keys

    \begin{tcolorbox}[breakable, size=fbox, boxrule=1pt, pad at break*=1mm,colback=cellbackground, colframe=cellborder]
\prompt{In}{incolor}{ }{\boxspacing}
\begin{Verbatim}[commandchars=\\\{\}]
\PY{n}{O} \PY{o}{=} \PY{n+nf}{optimize}\PY{p}{(}\PY{n}{LL}\PY{p}{,} \PY{n}{interval}\PY{o}{=}\PY{n+nf}{c}\PY{p}{(}\PY{l+m}{0}\PY{p}{,}\PY{l+m}{1}\PY{p}{)}\PY{p}{,}\PY{n}{maximum}\PY{o}{=}\PY{k+kc}{TRUE}\PY{p}{)}
\PY{n}{O}
\end{Verbatim}
\end{tcolorbox}

    The first key is called ``maximum'' and returns the maximum theta for
our log likelihood, in other words our maximum likelihood estiamtor of
the parameter \(\theta\). The second key returns the loglikelihood value
for that maximum \(\theta\).

    \begin{tcolorbox}[breakable, size=fbox, boxrule=1pt, pad at break*=1mm,colback=cellbackground, colframe=cellborder]
\prompt{In}{incolor}{ }{\boxspacing}
\begin{Verbatim}[commandchars=\\\{\}]
\PY{n}{mle} \PY{o}{=} \PY{n}{O}\PY{o}{\PYZdl{}}\PY{n}{maximum}
\PY{n}{mle}
\end{Verbatim}
\end{tcolorbox}

    \hypertarget{variance-of-our-mle}{%
\subsubsection{Variance of our MLE}\label{variance-of-our-mle}}

Let's explore the variability in our maximum likelihood estimator (MLE)
by simulating 1000 coin flips 5000 times, recording the MLE, and then
plotting a histogram of the MLE values.

    \begin{tcolorbox}[breakable, size=fbox, boxrule=1pt, pad at break*=1mm,colback=cellbackground, colframe=cellborder]
\prompt{In}{incolor}{51}{\boxspacing}
\begin{Verbatim}[commandchars=\\\{\}]
\PY{n}{Os} \PY{o}{=} \PY{n+nf}{rep}\PY{p}{(}\PY{l+m}{0}\PY{p}{,}\PY{l+m}{5000}\PY{p}{)}                      \PY{c+c1}{\PYZsh{} Create a vector of 5,000 zeros that will store our MLEs}
\PY{n+nf}{for}\PY{p}{(}\PY{n}{i} \PY{n}{in} \PY{l+m}{1}\PY{o}{:}\PY{l+m}{5000}\PY{p}{)}\PY{p}{\PYZob{}}                     \PY{c+c1}{\PYZsh{} Run a loop 5,000 times }
  \PY{n}{samples} \PY{o}{=} \PY{n+nf}{simulateNflips}\PY{p}{(}\PY{l+m}{1000}\PY{p}{,}\PY{l+m}{0.25}\PY{p}{)} \PY{c+c1}{\PYZsh{} Flip 1,000 coins with a prob of heads of 0.25}
  \PY{n}{LL} \PY{o}{=} \PY{n+nf}{function}\PY{p}{(}\PY{n}{theta}\PY{p}{)}\PY{p}{\PYZob{}}               \PY{c+c1}{\PYZsh{} build a function that incorporates the data above}
    \PY{n+nf}{return}\PY{p}{(}\PY{l+m}{\PYZhy{}1}\PY{o}{*}\PY{n+nf}{loglikelihood}\PY{p}{(}\PY{n}{theta}\PY{p}{,}\PY{n}{samples}\PY{p}{)}\PY{p}{)}
  \PY{p}{\PYZcb{}}
  \PY{n}{O} \PY{o}{=} \PY{n+nf}{optimize}\PY{p}{(}\PY{n}{LL}\PY{p}{,} \PY{n}{interval}\PY{o}{=}\PY{n+nf}{c}\PY{p}{(}\PY{l+m}{0}\PY{p}{,}\PY{l+m}{1}\PY{p}{)}\PY{p}{)}  \PY{c+c1}{\PYZsh{} Optimize!}
  \PY{n}{Os}\PY{p}{[}\PY{n}{i}\PY{p}{]} \PY{o}{=} \PY{n}{O}\PY{o}{\PYZdl{}}\PY{n}{minimum}                  \PY{c+c1}{\PYZsh{} Whoa, minimum? what happened here? Hint: look up three lines}
\PY{p}{\PYZcb{}}
\PY{n+nf}{hist}\PY{p}{(}\PY{n}{Os}\PY{p}{)}                             \PY{c+c1}{\PYZsh{}Plot a histogram of our 5,000 MLEs}
\end{Verbatim}
\end{tcolorbox}

    \begin{center}
    \adjustimage{max size={0.9\linewidth}{0.9\paperheight}}{Algorithms_II_Lab-02_files/Algorithms_II_Lab-02_32_0.png}
    \end{center}
    { \hspace*{\fill} \\}
    
    We see that the MLE is centered near \(\theta=0.25\) but does have some
variability around this \(\theta\) value. The shape of this histogram
likes awfully ``Normal''. This is because we learned in class that the
MLE has the following sampling distribution

\begin{align}
    \theta_{\text{MLE}} \sim \mathcal{N}\left( \theta , \mathcal{I}^{-1} \right)
\end{align} where \(\theta\) is the mean parameter and \(\mathcal{I}/N\)
the variance around the MLE.

We learned in class that the Fisher information for the Bernoulli
distribution for a single data point is

\begin{align}
    \mathcal{I}(\theta) = \frac{1}{\theta(1-\theta)}
\end{align}

and or \(N\) data points is

\begin{align}
    \mathcal{I}(\theta) = \frac{N}{\theta(1-\theta)}
\end{align}

Lets plot our empricial distribution of MLEs and also the above normal
distribution to see how well they line up.

    \begin{tcolorbox}[breakable, size=fbox, boxrule=1pt, pad at break*=1mm,colback=cellbackground, colframe=cellborder]
\prompt{In}{incolor}{55}{\boxspacing}
\begin{Verbatim}[commandchars=\\\{\}]
\PY{n}{fisherInfo} \PY{o}{=} \PY{n+nf}{function}\PY{p}{(}\PY{n}{MLE}\PY{p}{,}\PY{n}{samples}\PY{p}{)}\PY{p}{\PYZob{}}
  \PY{n}{N} \PY{o}{=} \PY{n+nf}{length}\PY{p}{(}\PY{n}{samples}\PY{p}{)}
  \PY{n+nf}{return}\PY{p}{(}\PY{n}{N}\PY{o}{/}\PY{n}{MLE}\PY{o}{*}\PY{p}{(}\PY{l+m}{1}\PY{o}{\PYZhy{}}\PY{n}{MLE}\PY{p}{)}\PY{p}{)} \PY{c+c1}{\PYZsh{} This is the Fisher info for N samples}
\PY{p}{\PYZcb{}}

\PY{n}{thetas} \PY{o}{=} \PY{n+nf}{seq}\PY{p}{(}\PY{l+m}{0}\PY{p}{,}\PY{l+m}{1}\PY{p}{,}\PY{l+m}{0.001}\PY{p}{)}
 
\PY{c+c1}{\PYZsh{} This returns the density of a Normal distribution with mean, and sd for all the x values}
\PY{n}{dens} \PY{o}{=} \PY{n+nf}{dnorm}\PY{p}{(}\PY{n}{x} \PY{o}{=} \PY{n}{thetas}\PY{p}{,} \PY{n}{mean} \PY{o}{=} \PY{l+m}{0.25}\PY{p}{,} \PY{n}{sd} \PY{o}{=} \PY{n+nf}{sqrt}\PY{p}{(}\PY{l+m}{1}\PY{o}{/}\PY{n+nf}{fisherInfo}\PY{p}{(}\PY{l+m}{0.25}\PY{p}{,}\PY{n}{samples}\PY{p}{)}\PY{p}{)}\PY{p}{)}

\PY{n+nf}{hist}\PY{p}{(}\PY{n}{Os}\PY{p}{,}\PY{n}{probability}\PY{o}{=}\PY{k+kc}{TRUE}\PY{p}{)} \PY{c+c1}{\PYZsh{} The options probability=TRUE estimates a probability density from your histogram}
\PY{n+nf}{lines}\PY{p}{(}\PY{n}{thetas}\PY{p}{,}\PY{n}{dens}\PY{p}{)}
\end{Verbatim}
\end{tcolorbox}

    \begin{center}
    \adjustimage{max size={0.9\linewidth}{0.9\paperheight}}{Algorithms_II_Lab-02_files/Algorithms_II_Lab-02_34_0.png}
    \end{center}
    { \hspace*{\fill} \\}
    

    % Add a bibliography block to the postdoc
    
    
    
\end{document}
